\chapter{Introduction}

%\begin{center}
%  {\it ``If could add an introductionary text here.''}
%  \vspace{1cm}
%\end{center}


\section{Introduction}
Observational studies and cosmological simulations have revealed a deep interconnection between galaxies and their central supermassive black hole (SMBH): they appear to coevolve and affect each other during their lives.
It has been reported several times that the SMBH mass (M$_{\rm BH}$) correlates with a number of galaxy properties, including bulge mass, velocity dispersion, and S\'{e}rsic index.{\bf explicitar que estas relaciones no son obvias, el BH es demasiado pequeño para influenciar directamente la dinámica de la galaxia.} %The relation with the velocity dispersion appears to the most fundamental relation so far \citep[e.g.,][]{2007ApJ...660..267B, 2016MNRAS.460.3119S, 2017MNRAS.466.4029S,2019MNRAS.485.1278S}.
By modeling the selection bias that leads to the M$_{\rm BH}$ and velocity dispersion $\sigma$ and M$_{\rm BH}$-galaxy bulge luminosity from optical or NIR bands $L$, \citet{2007ApJ...660..267B} have shown that the most fundamental relation is between M$_{\rm BH}-\sigma$ while the relation between M$_{\rm BH}- L$ is a consequence of a relation between $\sigma-L$ and is therefore more biased, as it predicts more massive BHs at a given luminosity.{\bf poner un par de oraciones explicando qué es el bias y por qué importa.} Also \citet{2017MNRAS.466.4029S} showed that M$_{\rm BH}-\sigma$ is the most fundamental in the scaling relations between black holes and galaxies by studying the residuals of the relations between M$_{\rm BH}$ and $\sigma$, S\'{e}rsic index $n$ and M$_{\rm bulge}$ in SDSS galaxies. Finally, Monte Carlo simulations have shown that a bias can be introduced when dynamically measuring M$_{\rm BH}$: the requirement that the black hole sphere of influence must be resolved to measure black hole masses leads to an increased M$_{\rm BH}-\sigma$ relation of at least a factor three \citep{2016MNRAS.460.3119S}.
%\citet{2007ApJ...660..267B} have modeled the selection bias that leads to the M$_{\rm BH}$ and velocity dispersion $\sigma$ and M$_{\rm BH}$-galaxy bulge luminosity $L$, from optical or NIR bands, and have shown that the most fundamental relation is between M$_{\rm BH}-\sigma$ while the relation between M$_{\rm BH}- L$ is a consequence of a relation between $\sigma-L$ and is therefore more biased, as it predicts more massive BHs at a given luminosity. 
%\citet{2016MNRAS.460.3119S} found through Monte Carlo simulations that a bias can be introduced when dynamically measuring M$_{\rm BH}$: the requirement that the black hole sphere of influence must be resolved to measure black hole masses leads to an increased M$_{\rm BH}-\sigma$ relation of at least a factor three. 
%\citet{2017MNRAS.466.4029S} lead a study in which they studied the residuals of the relations between M$_{\rm BH}$ and $\sigma$, S\'{e}rsic index $n$ and M$_{\rm bulge}$ in SDSS galaxies, and find that $\sigma$ is the most fundamental in the scaling relations between black holes and galaxies.

Likewise, the cosmic star formation rate density (SFRD) and black hole accretion rate densities (BHARD) share a similar evolution{\bf definir SRFS y BHARD, no solo el acronimo, pornle un par de oraciones explicando que se puede medir estas cantidades por unidad de volumn comovil, para distintos z y por lo tanto se puede estudiar su evolución. después lo que sale un poco más abajo y despues estas primeras líneas}: they reach a peak of activity at redshift $z\sim2$ and then decrease to the present epoch \citep{1998MNRAS.293L..49B, 2014MNRAS.439.2736D, 2014ARA&A..52..415M}.
The BHARD has been determined from the IR (up to $z\sim3$, using and \emph{Herschel} data, \citet{2014MNRAS.439.2736D}) and X-ray observations (up to $z\sim 6$ wide and deep surveys from XMM and Chandra, \citet{2018MNRAS.473.2378V}).
The history of star formation in galaxies throughout the life of the Universe has been thoroughly constrained in recent years: it can be estimated from the UV emission of the massive and short-lived stars, but a fraction of it is absorbed by the dust present in the galaxy and then re-emitted in the IR, therefore the SFRD has been studied through infrared data with \emph{Herschel}, ultraviolet data with \emph{Galaxy Evolution Explorer} (GALEX) \citep[see][for a review]{2014ARA&A..52..415M} and more recently complemented by sub-mm ALMA data, especially at high redshift up to $z\sim10$ \citep{2020ApJ...902..112B,2020A&A...643A...8G}. For both SFRD and BHARD the contribution from deep and wide field surveys has been of crucial importance.

Additionally, semianalytic models and hydrodynamic simulations show that a self-regulating mechanism is required between the star formation and the black hole accretion in order to reproduce local scaling relations {\bf definir scaling relations o al menos recordar lo de arriba y decir relacion entre BH mass y host galaxy properties.}\citep[see][for a review]{2015ARA&A..53...51S}. Since the pioneering models of galaxy evolution in a cold dark matter framework it was clear that there was an "overcooling problem": most of the gas should have cooled and condensed into stars by the present day while it is observed that less than 10\% of it is in the shape of stars, so some suppression of cooling and star formation had to be introduced. This could be as energy released by supernovae (Larson 1974, White \& Rees 1978, Dekel \& Silk 1986, White \& Frenk 1991) and the impact of this phenomenon would be important in regulating star formation in low mass galaxies. On the other hand, luminosity functions predict more high luminosity galaxies in the local universe then we observe, and this might be prevented from happening through AGN feedback{\bf definir AGN y feedback}. In fact simple calculations have shown that when the black hole becomes sufficiently massive ($> 10^7 M_\odot$) its Eddington {\bf definir Eddington, o decir simplemente maximum accretion luminosity, definir accretion, en realidad, antes de hablar de esto poner qué es un AGN y por qué brilla y por qué podría afectar a su galaxia, ponle un parrafito sobre esto.}luminosity becomes high enough that its winds could in principle blow out the entire galaxy {\bf el gas nomás espero} \citep{1998A&A...331L...1S}. This feedback mode, called quasar mode, prevents star formation by delivering momentum to the galactic gas and removing it. There is also a radio mode in which the BH is in a low accretion mode and heats the gas, thereby preventing it from collapsing and forming new stars \citep{2017NatAs...1E.165H}.
    
    %the SFRD of the Universe increased by a factor of about 10 since $z\sim8$, reached a peak at around $z\sim2$, and declined by a factor of 10 since then. 
    The decline of the SFRD at low redshift is thought to depend on the decreasing availability of cold gas that is required to form stars and accrete onto black holes \citep[e.g.,][]{2016MNRAS.458L..14F}. In addition, we know that most galaxies follow a sequence on the stellar mass ($M_*$) - star formation rate (SFR) plane with an almost linear slope{\bf un poco más explícito: algo como que se observa que mientras más masa estelar tiene la galaxia, más estrellas está formando por unidad de tiempo, con una relación casi lineal}. The so-called \emph{\textup{main-sequence}} of star-forming galaxies \citep{2004MNRAS.351.1151B, 2007ApJ...670..156D, 2007A&A...468...33E, 2007ApJ...660L..43N, 2014MNRAS.443...19R, 2017MNRAS.465.3390A} suggests that most of the cosmic star formation takes place through secular {\bf secular en oposición a qué, puedes explicar un poco más, como si hay dos corrientes posibles, que las estrellas se formen en bursts por mergers o algo o todo el tiempo por procesos seculares y que estas relaciones indican que es principalmente seculares} processes \citep{2011ApJ...739L..40R} because it has been observed at all redshifts and shows no evolution in its slope, but an increasing normalization with redshift \citep{2015A&A...581A..54T, 2015A&A...575A..74S, 2016ApJ...817..118T}: at earlier epochs, galaxies of a given stellar mass were forming more stars than in the local Universe.
    We also find that a number of galaxies lie substantially above and below the MS.
    Starburst galaxies are a small fraction of star-forming galaxies \citep[$\sim2\%$][]{2011ApJ...739L..40R}, and their SFRs and gas fractions are higher than those on the MS {\bf for the same stellar mass}. 
    Quiescent galaxies, on the other hand, lie below the main-sequence and therefore have little star formation. They have been shown to evolve from $z\sim1.8,$ where they have significant amounts of dust and gas ($\sim5-10\%$) but their star formation efficiency is low, to the local universe, where they are gas poor \citep{2018NatAs...2..239G}.
    
    Just like star formation, black hole accretion depends on the availability of cold gas. Simulations show that gas flows into galaxies, where it cools to eventually fuel star formation and black hole accretion \citep{2010MNRAS.407.1529H}. This fueling by gas is expected to be even more pronounced in starburst galaxies, many of which are likely to undergo a major merging event \citep{2005Natur.433..604D, 2008ApJS..175..356H}: During the early stages of galaxy merging, gas can efficiently cool and lose angular momentum, eventually feeding central star formation and black hole growth. 
    The black hole accretion is initially obscured by thick layers of dust, which are then possibly removed by its increasing radiation and momentum feedback, revealing the quasar. Eventually, the gas is consumed, the quasar luminosity fades rapidly, and the star formation episode ceases. This leaves a "red and dead" elliptical galaxy with no or very little star formation or black hole accretion \citep{2004ApJ...600..580G, 2006ApJ...650...42L}{\bf no me queda claro si esta evolución es una predicción, informada por sims o datos o ambos o está todo comprobado}.
    
    This picture might suggest that just like the majority of galaxies follow a main-sequence in the SFR-$M_*$ plane, a similar relation between the black hole accretion rate (BHAR) and the $M_*$ might exist: the more massive the galaxy, the higher the availability of inflowing gas for star formation and black hole accretion, which would mean that they both should correlate with stellar mass. In addition, galaxies offset from the main-sequence (starbursts and quiescents), might have a BHAR that varies accordingly with the gas that is typically available in that phase \citep{2019ApJ...877L..38R}. 
    
    In order to search for these potential correlations, many authors have used the X-ray luminosity of galaxies as a proxy of BHAR: X-rays are very energetic photons that are created very close to the central SMBH, and other contaminants in the host galaxies at these wavelengths, for example, emission from stellar processes or binary systems, are usually less powerful and not dominant \citep[e.g.][]{2015A&ARv..23....1B}. Nevertheless, the first studies that traced the instantaneous BHAR with X-ray flux failed at finding any BHAR-M$_*$ relation \citep{2009ApJ...696..396S, 2010A&A...518L..26S, 2012MNRAS.419...95M, 2012A&A...545A..45R, 2015ApJ...806..187A}. A lack of a correlation between SFR and BHAR does not by itself necessarily imply a lack of physical connection. It might arise, for example, from different duty cycles and variabilities that characterize the two processes.  
    Episodes of star formation last for several Gyr, while the SMBH duty cycles are believed to be very short, with accretion episodes of about $10^5$~yr and variability timescales that range from minutes to months.
    
    In order to constrain more robust and reliable BHAR, 
    it is thus necessary to average their growth rate over a long time interval. A very promising technique to achieve this goal consists of stacking X-ray images. 
    Stacking allows us to perform studies on mass-complete samples by averaging the count rates of the X-ray images in the optical positions of the galaxies, thus increasing the signal-to-noise ratio, and allowing us to reach fluxes well below the single-source detection threshold of the observations{\bf poner en otra oracion que el stacking se hace para galaxias con las mismas caracteristicas opticas, que quede claro que no se está mezclando todo}. Moreover, because the BHAR is a stochastic event, stacking large samples of galaxies in a given volume is equivalent to averaging the growth rate of all galaxies.{\bf pondría esto último primero, que es lo esencial para calcular el crecimiento promedio}
    Previous works indeed searched for a relation between M$_*$ and average BHAR by stacking X-ray images \citep[e.g.,][]{2012ApJ...753L..30M, 2015ApJ...800L..10R, 2017ApJ...842...72Y}, and the probability distribution of specific X-ray luminosity {\bf per galaxy? per bin?} with a maximum likelihood approach \citep{2012ApJ...746...90A, 2012MNRAS.427.3103B, 2018MNRAS.475.1887Y} and a Bayesian approach \citep{2018MNRAS.474.1225A}. All studies point toward a positive correlation between the BHAR and the M$_*$ for star-forming galaxies, very similar to the main-sequence of star-forming galaxies, with a slope close to unity, non-negligible redshift evolution, a positive slope for the BHAR-to-SFR ratio as a function of stellar mass, and indications of different behaviors for quiescent and starburst galaxies. 
    Thus far, no study has presented a complete analysis throughout all galaxy life phases by highlighting the evolution of the accretions and their ratio {\bf which ratio} throughout cosmic time. This is what we present here. {\bf pon más explicitamente que interrogante se va a responder con este anañisis más completo, aunque sea en varias oraciones. Hay teorías alternativas que se van a probar?} So far, only \citet{2019MNRAS.484.4360A} have shown that the fraction of active galactic nucleus (AGN) galaxies is higher below the main-sequence and in starbursts {\bf than in the main sequence and SF's?, aquí veremos qué pasa en las otras fases? para restingir qué cosa?}.
        
        We here characterize the evolution of the average BHAR for normal star-forming, quiescent, and starburst galaxies at $0.1<z<3.5$. This redshift interval encompasses the majority of the history of the Universe and contains two crucial epochs in its evolution: the peak of the star formation rate density and BHAR of the Universe at z$\sim$2, and their decline to the local Universe.
        We take advantage of the unique depth, area, and wavelength coverage of the COSMOS field, which allows us to select a mass-complete sample with large statistics out to very high redshifts for each galaxy phase. This is particularly important for starbursts, which are rare objects and require a large field in order to be found in good numbers for statistics. We apply the stacking technique to X-ray images from the \textit{Chandra} COSMOS-Legacy survey \citep{2016ApJ...819...62C} and combine the results with the actual X-ray detections in order to estimate the average X-ray luminosity and therefore average BHAR. We compare the evolution of the average BHAR with that of the average SFR. We show that these data confirm and extend previous claims about the evolution of the specific accretions and that the ratio of BHAR to SFR and M$_*$ are correlated. 
%More specifically, in Section \ref{sec:data} we introduce the parent data sample we used for this work, from which we select the sample as described in Section \ref{sec:sample}. In Section \ref{sec:method} we describe the method we used to estimate the average X-ray luminosities, BHAR and SFR. In Sections~\ref{sec:L_X} and~\ref{sec:BH_SF} we describe our results and discuss them. In Section~\ref{sec:specifics} we describe the derivation of specific BHAR and specific SFR and discuss the respective results. In Section~\ref{sec:M_vs_M} we compare our M$_\text{BH}$-M$_*$ relation to observational and simulation relations. Finally, in Section~\ref{sec:conclusions} we enumerate the conclusions of this work.
   
%Throughout this paper we use a \citet{2003PASP..115..763C} initial mass function (IMF) assuming a flat cosmology with $H_0=70$, $\Omega_\lambda=0.7$, $\Omega_0=0.3$.

%\begin{figure}[t]
%\begin{center}
%  \includegraphics[scale=0.5, angle=-90]{photo.png}
%  \caption{Here you can provide the caption.}
%    \label{and_the_label}
%  }
%\end{center}
%\end{figure}
