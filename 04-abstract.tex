\chapter*{Abstract}

In this Thesis we study how the accretion of the central supermassive black hole evolves in galaxies in all their star formation life phases and through a wide range of cosmic epochs. 

We take two complementary approaches. First, we perform a statistical study on a large sample of galaxies from the COSMOS field where we take advantage of X-ray Chandra data to estimate black hole accretions, via a combination of stacked data and individual detections, and compare them with their star formation properties, estimated from far-infrared emission combined with ultra-violet emission. Then, we use semi-empirical models to create galaxy mock catalogs onto which we perform an analogous analysis in order to pin down which parameters control the black holes' X-ray emission and its evolution.

%We study galaxies in bins of redshift and stellar mass and find a relation between average x.ray and m*, similar to the sfr-m* relation, at all redshifts and in the three galaxy types considered: star forming, quiescent and starburst. This relation has a decreasing trend in time which seems to be driven by a decrease in the average edd ration of galaxies with time and a slope that could arise from a combination of a superlinear Mbh-M* relation and a decreasing average edd ratio with mass. the lx-m* shows different normalization in te different galaxy life phases

We find a picture in which the bulk of the black hole and stellar masses are accreted in the star forming phase through secular processes, where the average black hole accretion follows a relation with stellar mass similar to the ``main sequence'', i.e. the relation between the star formation rate and the stellar mass followed by star forming galaxies, having a similar evolution in time but with a more efficient accretion at high stellar masses.
The starburst phase appears to have a significant enhancement of the SFR but a lesser impact on the black hole accretion, which has a samller enhancement especially at high redshift. 
Quiescent galaxies, on the other hand, undergo a significant decline in their star formation, while the black hole accretion is still noticeable.
This observed evolution of the X-ray luminosity with time and galaxy phase is compatible with a change in the average Eddington ratio but is mostly independent on the duty cycle.
We find a super-linear relation between black hole and stellar mass which, in order to reproduce the observations, should be combined with an average Eddington ratio that depends on stellar mass.
Our results point in the direction of galaxy downsizing, i.e. a fast accretion of the black hole and stellar mass at very high redshift for the most massive galaxies, followed by a steep decrease in accretion, while low-mass galaxies accrete their mass more slowly, with an accretion rate that decreases more slowly with time.

On a separate note, we study the gas distribution in a sample of local active galaxies, by analyzing their continuum and reflected X-ray light curves and reproducing the observed damping of the variations of the reflected component through Monte Carlo simulations.