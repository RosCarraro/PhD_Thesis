\chapter{A semi-analytic simulation point of view}

%\begin{center}
%  {\it ``If could add an introductionary text here.''}
%  \vspace{1cm}
%\end{center}

The connection between the black hole accretion, often traced by the X-ray luminosity (L$_{\rm X}$), and the star formation rate (SFR) in galaxies at all redshifts has been largely investigated, but still, the physical processes that drive their relation are not clear. 

We create mock catalogs of BHs by using semi-empirical relations, starting from samples of dark matter halos from large N-body simulations, to study the L$_{\rm X}-{\rm SFR}$ relation.  In detail, we assign at different redshifts, galaxies and black holes (BH) to dark matter halos via the most up-to-date empirical stellar-halo and stellar-BH mass relations and we assume an SFR depending only on stellar mass and redshift.

We find no causality between L$_{\rm X}$ and SFR, but we show, however, that the origin of this relation is in their common dependence on stellar mass $\log {\rm M}_*$. More specifically we see that the main driver of the slope of the relation is the stellar mass to black hole mass (M$_{\rm BH}-{\rm M}_*$) scaling relation, while the Eddington ratio distribution only affects its normalization. The duty cycle, on the other hand, does not play a significant role.


%\begin{figure}[t]
%\begin{center}
%  \includegraphics[scale=0.5, angle=-90]{photo.png}
%  \caption{Here you can provide the caption.}
%    \label{and_the_label}
%  }
%\end{center}
%\end{figure}
