\chapter{Conclusions}

%\begin{center}
%  {\it ``If could add an introductionary text here.''}
%  \vspace{1cm}
%\end{center}

In this thesis we have studied how accretion of the central supermassive black hole evolves in galaxies in all their star formation life phases and through a wide range of cosmic epochs. It is particularly interesting to look for a connection between these two accretion phenomena as they are both fed by cold gas, but they take place on very different galactic scales and it is not clear how the gas can lose angular momentum and funnel into the central black hole, so the two phenomena could in principle be completely independent. We have taken two complementary approaches, we have performed a statistical study on an observational sample from a deep and wide survey, the COSMOS survey, with an exceptional wavelength coverage (Chapter~\ref{ch:observations}), and we have adopted Semi-Empirical models to generate mock galaxy catalogs and thus better understand the physical parameters that shape the relations we found in the data (Chapter~\ref{ch:SEM}).

We have mainly focused on the average ${\rm L}_{\rm X}$ emission of galaxies divided in bins of stellar mass, the ${\rm L}_{\rm X}-{\rm M}_*$ relation, in three galaxy life phases: star forming, quiescent and starburst. We have seen that the ${\rm L}_{\rm X}-{\rm M}_*$ relation, which easily translates into a BHAR$-{\rm M}_*$ relation, follows a decreasing trend in time, probably as less gas is available in galaxies at later cosmic epochs thus causing a decrease in the average Eddington ratio, and that similarly to SFR, starbursts have a higher ${\rm L}_{\rm X}$ than star forming galaxies, and quiescents have a lower ${\rm L}_{\rm X}$. Our SEMs point in the direction that these evolutions in time and across galaxy type are driven by a change in the average Eddington ratio $\zeta_c$ which seems to rule the ${\rm L}_{\rm X}-{\rm M}_*$ normalization. Independent results from observations and models, and also our own from Sec.~\ref{sec:M_vs_M}, suggest that the ${\rm M}_{\rm BH}-{\rm M}_*$ relation is constant with redshift but with a steeper slope than the one that would return our observed ${\rm L}_{\rm X}$ slope, therefore implying an average Eddington ratio $\zeta_c$ decreasing with increasing stellar mass.

We have seen that main-sequence and quiescent galaxies share similar ratios of BHAR and SFR at all probed cosmic epochs, suggesting that the two processes are linked together throughout different galaxy phases. This ratio indicates that at fixed ${\rm M}_{\rm BH}/{\rm M}_*$, a similar BHAR/SFR ratio as the one observed in star forming and quiescent galaxies, would be induced by a proportional decline in characteristic Eddington ratio $\zeta_c$ and specific SFR within a bin of stellar mass. Analogously, the significantly lower BHAR/SFR in starbursts with respect to quiescent/star forming galaxies would be naturally interpreted as a proportionally higher specific SFR and roughly constant or slightly higher $\zeta_c$ in these young, gas rich systems.

Our results on sBHAR and sSFR show similar evolutions and signs of downsizing in all galaxy types and at all explored redshifts, where we define downsizing as the fact that more massive galaxies accreted most of their black hole mass and of the stellar mass at very early cosmic epochs and their accretion decreased fast and steeply, whereas low-mass galaxies have accreted their mass more slowly, but their accretion rate decreased more slowly with time. Downsizing can also be spotted from our mock catalogs from SEMs, that suggest that more massive galaxies have lower characteristic Eddington ratios $\zeta_c$, and is mostly apparent in quiescent galaxies.

Our work points in the direction of a co-evolution of star formation and black hole accretion in galaxies across cosmic time where the bulk of the black hole and stellar masses is accreted in galaxies during the main-sequence phase through secular processes.
Both accretions follow similar evolutionary patterns which appear to be driven by cold gas availability which affects star formation and the characteristic Eddington ratio $\zeta_c$, even though the magnitude of this effect is different in both accretions: the starburst phase seems to be accompanied by a substantial enhancement of the SFR but a limited enhancement of the BHAR, especially at high redshifts. On the other hand, the quiescent phase seems characterized by very little star formation but still a significant residual black hole accretion. The final evolutionary picture that emerges from this work is one in which the host galaxy starts off from a main-sequence or even starburst, gas-rich phase, evolving at an almost constant (specific) SFR and then gradually switches off its black hole accretion and star formation due to internal gas consumption, thus gradually reducing the SFR and BHAR.

\vspace{10pt}
On a distinct approach, in Chapter~\ref{ch:gas_distribution} we have studied the gas distribution around the AGN by simulating the variability of X-ray emissions from the corona and from the reprocessing gas in a sample of local galaxies. We used a simple model which assumed the reprocessing gas to be arranged in a spherical thin shell around the active galactic nucleus, which acts by delaying and smoothing out the variations of the continuum emission from the corona. This very simple model, together with sparse light curves from archival data allowed to place constraints for 17 out of 33 sources, constraints which could be useful in order to design future monitoring surveys. We also found a correlation with a large scatter between the size of the reprocessing gas and the mass of the black hole, suggesting that the mass of the BH plays a role in defining the gas arrangement around it, but also that our model may be too simple for some of our galaxies, and that the size of the reprocessor could depend on other physical factors like more complex geometries and/or AGN type dependent emission imprints. 